\documentclass[12pt,utf8,notheorems,compress,t]{beamer}

\usepackage[ngerman]{babel}

\usepackage{ragged2e}

\usepackage[protrusion=true,expansion=false]{microtype}

\setlength\parskip{\medskipamount}
\setlength\parindent{0pt}

\title{Dokumentverwaltung \\ mit Git und GitHub}
\author[LUGA]{Ingo Blechschmidt \\ \texttt{<iblech@web.de>}}
\date{Linux User Group Augsburg e.\,V. \\ 7. Januar 2015}

\usetheme{Warsaw}
\usecolortheme{seahorse}
%\usefonttheme{serif}
%\usepackage{mathpazo}
\usepackage{kurier}
\useinnertheme{rectangles}
\setbeamercovered{invisible}

\setbeamercolor{structure}{fg=purple}

% aus beamerouterthemeshadow.sty
\makeatletter
\setbeamertemplate{frametitle}
{%
  \nointerlineskip%
  \vskip-2pt%
  \hbox{\leavevmode
    \advance\beamer@leftmargin by -12bp%
    \advance\beamer@rightmargin by -12bp%
    \beamer@tempdim=\textwidth%
    \advance\beamer@tempdim by \beamer@leftmargin%
    \advance\beamer@tempdim by \beamer@rightmargin%
    \hskip-\Gm@lmargin\hbox{%
      \setbox\beamer@tempbox=\hbox{\begin{minipage}[b]{\paperwidth}%
          \vbox{}\vskip-.75ex%
          \leftskip0.3cm%
          \rightskip0.3cm plus1fil\leavevmode
          \centering%
          \insertframetitle%
          \ifx\insertframesubtitle\@empty%
            \strut\par%
          \else
            \par{\usebeamerfont*{framesubtitle}{\usebeamercolor[fg]{framesubtitle}\insertframesubtitle}\strut\par}%
          \fi%
          \nointerlineskip
          \vbox{}%
          \end{minipage}}%
      \beamer@tempdim=\ht\beamer@tempbox%
      \advance\beamer@tempdim by 2pt%
      \begin{pgfpicture}{0pt}{0pt}{\paperwidth}{\beamer@tempdim}
        \usebeamercolor{frametitle right}
        \pgfpathrectangle{\pgfpointorigin}{\pgfpoint{\paperwidth}{\beamer@tempdim}}
        \pgfusepath{clip}
        \pgftext[left,base]{\pgfuseshading{beamer@frametitleshade}}
      \end{pgfpicture}
      \hskip-\paperwidth%
      \box\beamer@tempbox%
    }%
    \hskip-\Gm@rmargin%
  }%
  \nointerlineskip
    \vskip-0.2pt
    \hbox to\textwidth{\hskip-\Gm@lmargin\hskip-\Gm@rmargin}
    \vskip-2pt
}
\makeatother

\setbeamertemplate{title page}[default][colsep=-1bp,rounded=false,shadow=false,bg=white]

\setbeamertemplate{navigation symbols}{}

\newcommand{\backupstart}{
  \newcounter{framenumberpreappendix}
  \setcounter{framenumberpreappendix}{\value{framenumber}}
}
\newcommand{\backupend}{
  \addtocounter{framenumberpreappendix}{-\value{framenumber}}
  \addtocounter{framenumber}{\value{framenumberpreappendix}} 
}

\newcommand*\oldmacro{}%
\let\oldmacro\insertshorttitle%
\renewcommand*\insertshorttitle{%
  \oldmacro\hfill\insertframenumber\,/\,\inserttotalframenumber\hfill}

\newcommand{\hil}[1]{{\usebeamercolor[fg]{item}{\textbf{#1}}}}

\newcommand{\img}[2]{\begin{center}\includegraphics[scale=#1]{#2}\end{center}}
\newcommand{\imageslide}[3]{\frame{\frametitle{#1}\img{#2}{#3}}}

\newcommand{\floatbox}[3]{%
  \raisebox{0pt}[0pt][0pt]{%
    \begin{picture}(0,0)(#1,#2)#3\end{picture}%
  }%
}

\begin{document}

\frame{\titlepage}

\section{Versionskontrolle}
\frame{\frametitle{Das 21. Jahrhundert}
  \centering
  \vfill
  
  \begin{minipage}{0.6\textwidth}
    \texttt{Bericht.odt} \\
    \texttt{Bericht\_2015-01-05.odt} \\
    \texttt{Bericht\_final.odt} \\
    \texttt{Bericht\_final\_nach\_Korrektur.odt} \\
    \texttt{Bericht\_ganz\_fertig.odt} \\
    \texttt{Bericht\_Abgabeversion.odt} \\
    \texttt{Bericht\_Abgabeversion\_korr.odt}
  \end{minipage}
  \vfill
  
  \hil{Das muss nicht sein!}
  \vfill
}

\frame{\frametitle{Was ermöglichen Versionskontrollsysteme?}
  \begin{itemize}
    \item Verhinderung von Dateinamenswirrwarr
    \item Überblick über die zeitliche Entwicklung eines Texts
    \item Rückkehr zu früheren Dateiversionen
    \item Paralleles Arbeiten an verschiedenen Textvarianten
    \item Einfache Zusammenarbeit im Team
    \item Analysemöglichkeiten, um Bugs einzugrenzen \\ (bei Software-Projekten)
  \end{itemize}
}

\frame{\frametitle{Grundlegende Konzepte}
  \begin{itemize}
    \item Das \hil{Arbeitsverzeichnis} enthält nur eine bestimmte Version (meistens die
    neueste).
    \item Die vollständige Versionshierarchie liegt in einem separaten \hil{Repository}.
    \item Mehrere Änderungen werden zu einzelnen \hil{Commits} zusammengefasst.
    \item Eigene Commits können an Teammitglieder \hil{gepusht} werden.
    \item Umgekehrt \hil{pullt} man fremde Änderungen.
  \end{itemize}
}


\section{Git}

\frame{\frametitle{Auftritt Git}
  \begin{itemize}
    \item Git ist ein modernes, dezentrales Versionskontrollsystem.
    \item Gestartet von Linus Torvalds 2005.
    \item De-facto-Standard in der Open-Source-Welt.
    \item Alternativen: Bazaar, Darcs, Mercurial, Subversion
  \end{itemize}

  Die wichtigsten Befehle, \hil{Live-Demo}:
  \begin{itemize}
    \item \texttt{git init} oder \texttt{git clone}
    \item \texttt{git add}
    \item \texttt{git commit}
    \item \texttt{git pull} oder \texttt{git pull --rebase}
    \item \texttt{git push}
    \item \texttt{gitk} oder \texttt{gitg}
  \end{itemize}
}


\section{GitHub}

\frame{\frametitle{GitHub}
  \begin{itemize}
    \item Viele Open-Source-Projekte verwalten ihre Repositories auf GitHub.
    \item Gestartet 2008.
    \item Kostenlos für öffentliche Projekte.
    \item Starke soziale Komponente: Pull Requests.
  \end{itemize}

  \begin{center}\hil{Live-Demo}\end{center}
}

\end{document}
